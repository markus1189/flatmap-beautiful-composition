 \documentclass[aspectratio=169]{beamer}

% Must be loaded first
\usepackage{tikz}
\usetikzlibrary{shapes,arrows,positioning,fit,calc}

\usepackage[utf8]{inputenc}
\usepackage{textpos}

% Font configuration
\usepackage{fontspec}

\usepackage[listings, minted]{tcolorbox}
\usepackage{smartdiagram}
\usepackage[export]{adjustbox}

\input{font.tex}

% Tikz for beautiful drawings
\usetikzlibrary{mindmap,backgrounds}
\usetikzlibrary{arrows.meta,arrows}
\usetikzlibrary{shapes.geometric}

% Minted configuration for source code highlighting
\usepackage{minted}
\setminted{highlightcolor=orange!50, linenos}
\setminted{style=lovelace}

\usepackage[listings, minted]{tcolorbox}
\tcbset{left=6mm}

% Use the include theme
\usetheme{codecentric}

% Metadata
\title{Beautiful Composition}
\author{Markus Hauck (@markus1189)}

% The presentation content
\begin{document}

\begin{frame}[noframenumbering,plain]
  \titlepage{}
\end{frame}

\section{Introduction}\label{sec:introduction}

\begin{frame}
  \begin{center}
    {\Huge Composition\\}
    (how to do more with less)
  \end{center}
\end{frame}

\begin{frame}
  \frametitle{Composition}
  \begin{itemize}
  \item most ``patterns'' in OO don't compose well
  \item definitely not: Design Patterns
  \item \texttt{SimpleBeanFactoryAwareAspectInstanceFactory}
  \item luckily, many concepts in FP compose naturally
  \item powerful building blocks (feels like LEGO)
  \end{itemize}
  \vfill
  \url{https://github.com/markus1189/flatmap-beautiful-composition}
\end{frame}

\begin{frame}[fragile]
  \frametitle{The Paper}
  \begin{columns}
    \begin{column}{0.4\textwidth}
      \begin{center}
        \small
        Gibbons, Jeremy, and Bruno C. D. S. Oliveira. "The essence of
        the iterator pattern." Journal of functional programming 19.3-4
        (2009): 377-402.
      \end{center}

    \end{column}
    \begin{column}{0.7\textwidth}
      \begin{center}
        \includegraphics[height=0.8\textheight, valign=c]{static-images/eip.png}
      \end{center}
    \end{column}
  \end{columns}
\end{frame}

\begin{frame}[fragile]
  \frametitle{Content}
  \begin{center}
    \resizebox{!}{0.8\textheight}{
      \smartdiagram[bubble diagram]{
        Content,
        2 Monoids,
        1 Introduction,
        5 Conclusion,
        4 Streaming,
        3 Applicatives
      }
    }
  \end{center}
\end{frame}

\begin{frame}
  \frametitle{Beautiful Composition \textemdash{} Themes}
  \begin{center}
    {
      \smartdiagramset{
        set color list={beamer@centricgreen!80,beamer@codeblue!80,red!50},
        module minimum width=3cm,
        text width=2.8cm
      }
      \smartdiagram[circular diagram]{
        Principle of Least Power,
        Composable Abstractions
      }
    }
  \end{center}
  \begin{itemize}
  \item design: the principle of least power
  \item important property of an abstraction: composition
  \end{itemize}
\end{frame}

\begin{frame}
  \frametitle{The Principle Of Least Power}
  \begin{itemize}
  \item something I notice often
  \end{itemize}
  \begin{tcolorbox}[
    fonttitle=\sffamily\bfseries,
    colbacktitle=black,
    colframe=black,
    coltitle=beamer@centricgreen,
    title=A Programmer
    ]
    ``I just use flatMap and be done with it''
  \end{tcolorbox}
  \begin{itemize}
  \item but: there are benefits of using sth with less power
  \item parallel execution with \texttt{Applicatives}
  \item less power equals more possibilities (Either vs. Validated)
  \end{itemize}
\end{frame}

\begin{frame}
  \frametitle{The Principle Of Least Power}
    \begin{tcolorbox}[
      fonttitle=\sffamily\bfseries,
      colbacktitle=black,
      colframe=black,
      coltitle=beamer@centricgreen,
      title=The Principle
      ]
      ``\ldots{} picking not the most powerful solution but the least powerful''
  \end{tcolorbox}
  \begin{itemize}
  \item \url{https://www.w3.org/DesignIssues/Principles.html}
  \item same is valid for abstractions from functional programming
  \item if you don't need Monad, don't require it
  \item \texttt{Applicative} vs \texttt{Functor}, \texttt{Monoid} vs \texttt{Semigroup}, etc.
  \end{itemize}
  \vfill
  \begin{quote}
    The reason for this is that the \textbf{less powerful} the language, \textbf{the
      more} you can do with the \textbf{data} stored\ldots
  \end{quote}
\end{frame}

\begin{frame}
  \frametitle{Composition}
  \begin{itemize}
  \item abstractions help to hide details and reason at a higher level
  \item real power comes when those abstractions compose (otherwise throwaway islands)
  \item most ``classic'' GoF Design Patterns in OO fail to compose
  \item many ``mathematical'' abstractions in FP compose naturally
  \item example: inductive monoid, product of monoids (same for Applicatives, \ldots{})
  \end{itemize}
\end{frame}

\begin{frame}[fragile]
  \frametitle{The Case Study}
  \begin{itemize}
  \item we want to analyze text
  \item collect metrics
  \item single traversal
  \item in essence a simple version of the GNU \texttt{wc} commandline tool
  \end{itemize}
  \begin{minted}{sh}
    bash> wc moby-dick.txt
    21206  208425 1193382 moby-dick.txt
  \end{minted}
  \begin{itemize}
  \item \texttt{lines}, \texttt{words} and \texttt{chars}
  \end{itemize}
\end{frame}

\begin{frame}
  \frametitle{Not For The Faint Of Heart}
  \begin{center}
    \includegraphics{static-images/warning.png}
    \includegraphics{static-images/see-no-evil.png}
    \includegraphics{static-images/hear-no-evil.png}
    \includegraphics{static-images/warning.png}
  \end{center}
  \vfill

  Gibbons, Jeremy, and Bruno C. D. S. Oliveira. ``The essence of the
  iterator pattern.''
\end{frame}

\begin{frame}[fragile]
  \frametitle{A First Solution}
  \inputminted[fontsize=\small]{scala}{snippets/imperative-wc.scala}
\end{frame}

\begin{frame}
  \frametitle{Problems}
  \begin{itemize}
  \item hard to understand
  \item hard to change
  \item mixed logic of the three metrics
  \item run only certain metrics? (e.g., only words)
  \item hardcoded control flow of char-by-char iteration
  \item very little abstraction
  \item can we do better?
  \end{itemize}
\end{frame}

\begin{frame}
  \frametitle{Metrics}
  \begin{itemize}
  \item number of chars (count each char)
  \item number of lines (count newline characters)
  \item number of words (harder, because it is context sensitive)
  \item open for extension (closed for modification)
  \end{itemize}
\end{frame}

\section{Monoids}\label{sec:monoids}

\begin{frame}
  \frametitle{Monoids}
  \begin{center}
    \huge
    Monoids
  \end{center}
\end{frame}

\begin{frame}
  \frametitle{Get Our FP Tools!}
  \begin{center}
    \includegraphics{static-images/toolbox.png}
    \includegraphics{static-images/construction-worker.png}
  \end{center}
\end{frame}

\begin{frame}[fragile]
  \frametitle{Monoids}
  \begin{itemize}
  \item Quick Recap: Monoids
  \item binary method \texttt{combine} and nullary method \texttt{empty}
    \vspace{1cm}
    \begin{minted}{scala}
trait Monoid[A] {
  def combine(x: A, y: A): A
  def empty: A
}
//infix combine operator: x |+| y
    \end{minted}
  \end{itemize}
\end{frame}

\begin{frame}
  \frametitle{Monoids: Basic Idea}
  \begin{itemize}
  \item basic approach: iterate over all \texttt{Char} and accumulate using a monoid
  \end{itemize}
\end{frame}

\begin{frame}[fragile,t]
  \frametitle{Monoids: Counting Chars}
  \begin{itemize}
  \item counting chars is easy, use \texttt{(Int, +)} as a Monoid
  \item count 1 (\texttt{combine} 1) for every character
  \end{itemize}
  \vspace{1cm}
  \begin{tikzpicture}[node distance = 8mm, auto]
    \tikzstyle{box} = [rectangle, draw, text centered, rounded corners, minimum width=6mm, minimum height=6mm]
    \tikzstyle{mon} = [rectangle, draw, text centered, minimum width=25mm, minimum height=6mm, fill=red!10]
    \tikzstyle{line} = [draw, -latex']

    \node [box, draw=none] (a0) {};
    \node [box, draw=none, left of=a0] (dummy) {};

    \foreach \char [count=\pos, evaluate=\pos as \i using int(\pos-1)] in {h,e,l,l,o,\textbackslash{n},w,o,r,l,d,!,\textbackslash{n}} {
      \ifthenelse{\isodd{\pos}}{
        \node [box, fill=beamer@codeblue!70, right of=a\i] (a\pos) {\char};
      }{
        \node [box, fill=beamer@centricgreen!70, right of=a\i] (a\pos) {\char};
      }
      \only<\pos>{
        \node [mon, below of=a\pos, yshift=-5mm] (b\pos) {\i\ + 1 = \pos};
        \path [line] (b\pos) -- (a\pos);
      }
    }
  \end{tikzpicture}

  \vfill
  \only<13>{
    \begin{itemize}
    \item so the result is 13 chars in total
    \end{itemize}
  }
\end{frame}

\begin{frame}[fragile, t]
  \frametitle{Monoids: Counting Lines}
  \begin{itemize}
  \item to count lines, use again \texttt{(Int, +)} as a Monoid
  \item but \textbf{only} count 1 if the character is a \textbackslash{n}
  \end{itemize}
  \vspace{1cm}
  \begin{tikzpicture}[node distance = 8mm, auto]
    \tikzstyle{box} = [rectangle, draw, text centered, rounded corners, minimum width=6mm, minimum height=6mm]
    \tikzstyle{mon} = [rectangle, draw, text centered, minimum width=25mm, minimum height=6mm, fill=red!10]
    \tikzstyle{line} = [draw, -latex']

    \node [box, draw=none] (a0) {};
    \node [box, draw=none, left of=a0] (dummy) {};

    \foreach \char [count=\pos, evaluate=\pos as \i using int(\pos-1)] in {h,e,l,l,o,\textbackslash{n},w,o,r,l,d,!,\textbackslash{n}} {
      \ifthenelse{\isodd{\pos}}{
        \node [box, fill=beamer@codeblue!70, right of=a\i] (a\pos) {\char};
      }{
        \node [box, fill=beamer@centricgreen!70, right of=a\i] (a\pos) {\char};
      }
      \only<\pos>{
        \ifthenelse{\pos > 5}{
          \ifthenelse{\pos > 12}{
            \node [mon, below of=a\pos, yshift=-5mm] (b\pos) {1\ + 1 = 2};
          }{
            \ifthenelse{\pos = 6}{
              \node [mon, below of=a\pos, yshift=-5mm] (b\pos) {0\ + 1 = 1};
            }
            {
              \node [mon, below of=a\pos, yshift=-5mm] (b\pos) {1\ + 0 = 1};
            }
          }
        }{
          \node [mon, below of=a\pos, yshift=-5mm] (b\pos) {0 + 0 = 0};
        }
        \path [line] (b\pos) -- (a\pos);
      }
    }
  \end{tikzpicture}

  \only<13>{
    \begin{itemize}
    \item we counted 2 lines in total
    \end{itemize}
  }
\end{frame}

\begin{frame}
  \frametitle{Composing Monoids}
  \begin{itemize}
  \item now: count chars \textbf{and} lines
  \item for multiple metrics, do multiple passes?!
  \item no \textemdash{} because monoids \textbf{compose}
    \begin{itemize}
    \item inductive: monoid + base monoid
    \item product: tuple of monoids
    \end{itemize}
  \end{itemize}
\end{frame}

\begin{frame}[fragile]
  \frametitle{Monoid Composition \textemdash{} Induction}
  \begin{itemize}
  \item some Monoids are based inductively on others
  \end{itemize}
  \begin{minted}{scala}
def optionMonoid[A: Monoid] = new Monoid[Option[A]] { /*...*/ }
  \end{minted}
  \begin{itemize}
  \item Option, Future, IO, Task, \ldots{}
  \item the Option-Monoid works like this:
  \end{itemize}
  \begin{center}
    \vspace{5mm}
    \begin{minted}{scala}
None |+| y === y
x |+| None === x
Some(x) |+| Some(y) === Some(x |+| y)
    \end{minted}
  \end{center}
\end{frame}

\begin{frame}[fragile]
  \frametitle{Monoid Composition \textemdash{} Option And Stopwords}
  \begin{itemize}
  \item as an example: filter out (don't count) stopwords
  \item stopwords = most common words that are not interesting (``the'', ``a'', \ldots{})
  \item idea: if it is a stopword, use None, otherwise regular count with Some
  \end{itemize}
\end{frame}

\begin{frame}[fragile]
  \frametitle{Monoid Composition \textemdash{} Option And Stopwords}
  \begin{itemize}
  \item assuming both ``is'' and ``a'' are classified as stopwords:
  \end{itemize}
  \vspace{1cm}
  \begin{tikzpicture}[node distance = 15mm, auto]
    \tikzstyle{box} = [rectangle, draw, text centered, rounded corners, minimum width=6mm, minimum height=6mm]
    \tikzstyle{mon} = [rectangle, draw, text centered, minimum width=25mm, minimum height=6mm, fill=red!10]
    \tikzstyle{line} = [draw, -latex']

    \node [box, draw=none] (a0) {};
    \node [box, draw=none, left of=a0] (dummy) {};

    \foreach \word [count=\pos, evaluate=\pos as \i using int(\pos-1)] in {this,is,a,test,text} {
      \ifthenelse{\isodd{\pos}}{
        \node [box, fill=beamer@codeblue!70, right of=a\i] (a\pos) {\word};
      }{
        \node [box, fill=beamer@centricgreen!70, right of=a\i] (a\pos) {\word};
      }
      \only<\pos>{
        \ifthenelse{\pos = 2 \OR \pos = 3}{
          \node [mon, below of=a\pos, yshift=-5mm] (b\pos) {Some(1)\ |+| None = Some(1)};
        }{
          \ifthenelse{\pos = 1}{
            \node [mon, below of=a\pos, yshift=-5mm] (b\pos) {None\ |+| Some(1) = Some(1)};
          }{
            \ifthenelse{\pos=4}{
              \node [mon, below of=a\pos, yshift=-5mm] (b\pos) {Some(1)\ |+| Some(1) = Some(2)};
            }{
              \node [mon, below of=a\pos, yshift=-5mm] (b\pos) {Some(2)\ |+| Some(1) = Some(3)};
            }
          }
        }
        \path [line] (b\pos) -- (a\pos);
      }
    }
  \end{tikzpicture}
  \vspace{1cm}
  \begin{itemize}
  \item<5> count without stopwords is 3
  \end{itemize}
\end{frame}

\begin{frame}
  \frametitle{Monoid Composition \textemdash{} Option And Stopwords}
  \begin{itemize}
  \item use \texttt{Option} plus \texttt{Max,Min} to get longest/shortest non-stopword
  \item more options:
    \begin{itemize}
    \item don't count chars like \texttt{!?,.} etc.\ using Option again
    \item use \texttt{Future/Task/IO} to get parallelism
    \item and sooo much more
    \end{itemize}
  \end{itemize}
\end{frame}

\begin{frame}
  \frametitle{Monoid Composition \textemdash{} Induction}
  \begin{itemize}
  \item base instance does not have to be a Monoid
  \item using \texttt{Option} we can lift any \texttt{Semigroup}
  \item \texttt{empty} becomes \texttt{None}
  \item useful for e.g. \texttt{Max} and \texttt{Min} to represent lower/upper bound
  \end{itemize}
\end{frame}

\begin{frame}[fragile]
  \frametitle{Monoid Composition \textemdash{} Tuple}
  \begin{itemize}
  \item if \texttt{A} and \texttt{B} have a Monoid instance, so does \texttt{(A,B)}
  \end{itemize}
  \begin{minted}{scala}
def tupleMonoid[A, B]: Monoid[(A, B)] = new Monoid[(A, B)] {
  def empty = (Monoid[A].empty, Monoid[B].empty)

  def combine(x: (A, B), y: (A, B)) = (x._1 |+| y._1, x._2 |+| y._2)
}
  \end{minted}
  \begin{itemize}
  \item combine the two \texttt{A}'s and the two \texttt{B}'s
  \item we can fuse our two metrics!
  \end{itemize}
\end{frame}

\begin{frame}[fragile, t]
  \frametitle{Multiple Monoids \textemdash{} One Traversal}
  \vspace{15mm}
  \begin{tikzpicture}[node distance = 8mm, auto]
    \tikzstyle{box} = [rectangle, draw, text centered, rounded corners, minimum width=6mm, minimum height=6mm]
    \tikzstyle{mon} = [rectangle, draw, text centered, minimum width=25mm, minimum height=6mm, fill=red!10]
    \tikzstyle{line} = [draw, -latex']

    \node [box, draw=none] (a0) {};
    \node [box, draw=none, left of=a0] (dummy) {};

    \foreach \char [count=\pos, evaluate=\pos as \i using int(\pos-1)] in {h,e,l,l,o,\textbackslash{n},w,o,r,l,d,!,\textbackslash{n}} {
      \ifthenelse{\isodd{\pos}}{
        \node [box, fill=beamer@codeblue!70, right of=a\i] (a\pos) {\char};
      }{
        \node [box, fill=beamer@centricgreen!70, right of=a\i] (a\pos) {\char};
      }
      \only<\pos>{
        \ifthenelse{\pos > 5}{
          \ifthenelse{\pos > 12}{
            \node [mon, below of=a\pos, yshift=-5mm] (b\pos) {1\ + 1 = 2};
          }{
            \ifthenelse{\pos = 6}{
              \node [mon, below of=a\pos, yshift=-5mm] (b\pos) {0\ + 1 = 1};
            }
            {
              \node [mon, below of=a\pos, yshift=-5mm] (b\pos) {1\ + 0 = 1};
            }
          }
        }{
          \node [mon, below of=a\pos, yshift=-5mm] (b\pos) {0 + 0 = 0};
        }

        \node [mon, below of=b\pos, yshift=2mm] (c\pos) {\i\ + 1 = \pos};
        \node [mon, below of=c\pos, yshift=2mm] (d\pos) {\ldots{}};
        \path [line] (b\pos) -- (a\pos);
      }
    }
  \end{tikzpicture}
  \only<13>{
    \begin{itemize}
    \item result: \texttt{2} lines and \texttt{13} chars
    \end{itemize}
  }
\end{frame}

\begin{frame}
  \frametitle{More Monoids}
  \begin{itemize}
  \item for \texttt{wc} we have two of our three target metrics
  \item but, we can do a lot more than that!
    \begin{itemize}
    \item find longest word
    \item count occurrences by word
    \item average word length (as own monoid or derive)
    \end{itemize}
  \item map of key to value (for any monoid as value)
  \end{itemize}
\end{frame}

\begin{frame}[fragile]
  \frametitle{Wordcount (Monoids, Chars)}
  \only<1>{\inputminted[fontsize=\footnotesize]{scala}{snippets/wc-monoid-char.scala}}
  \only<2>{\inputminted[highlightlines={11}, fontsize=\footnotesize]{scala}{snippets/wc-monoid-char.scala}}
\end{frame}

\begin{frame}
  \frametitle{Wordcount (Monoids, Chars)}
  \begin{itemize}
  \item the problem: we can't detect words
  \item we require some form of memory or state
  \item alternative idea: pre-split the text into words
  \end{itemize}
\end{frame}

\begin{frame}[fragile]
  \frametitle{Wordcount (Monoids, Strings)}
  \only<1>{\inputminted[fontsize=\footnotesize]{scala}{snippets/wc-monoid-string.scala}}
  \only<2>{\inputminted[highlightlines={13}, fontsize=\footnotesize]{scala}{snippets/wc-monoid-string.scala}}
\end{frame}

\begin{frame}
  \frametitle{Monoids \textemdash{} Review}
  \begin{itemize}
  \item stuck for now, but:
  \item composes very well from small blocks
  \item easy to extend using custom metrics
  \item \texttt{Option} can lift any Semigroup
  \item works beautifully with everything that can be folded
  \end{itemize}
  \vspace{1cm}
  \inputminted[fontsize=\small]{scala}{snippets/foldable-def.scala}
\end{frame}

\section{Applicatives}\label{sec:applicatives}

\begin{frame}
  \frametitle{Applicative}
  \begin{center}
    { \huge Applicatives\\ }
    (a.k.a. ``the bigger hammer'')
  \end{center}
\end{frame}

\begin{frame}[fragile]
  \frametitle{From Monoids To Applicatives}
  \begin{itemize}
  \item also called \textbf{Monoidal} Functors
  \item ``monoidal'' in the effects (pure = ``empty'' effect, \texttt{<*>} = ``combine effects'')
  \end{itemize}
  \vspace{5mm}
  \inputminted[fontsize=\small]{scala}{snippets/applicative-def.scala}
\end{frame}

\begin{frame}[fragile]
  \frametitle{Applicative as Effect Monoids}
  \begin{itemize}
  \item \texttt{Applicatives} adds richer structure, \textbf{monoidal} in effects
  \item \texttt{pure} is like \texttt{empty} \textit{for effects}
  \item \texttt{product}\footnote{from Semigroupal} is like \texttt{combine} \textit{for effects}
  \end{itemize}
  \begin{minted}{scala}
def empty      : F
def pure(x: A) : F[A]

def combine( x: F,     y: F)   : F
def product(fa: F[A], fb: F[B]): F[(A, B)]
  \end{minted}
\end{frame}

\begin{frame}
  \frametitle{No Monoids? \includegraphics[width=6mm, valign=c]{static-images/scream.png}}
  \begin{itemize}
  \item but wait a moment, was all we learned about Monoids for nothing?!
  \item rejoice, we can reuse \textbf{everything} we have until now
  \item in two ways
    \begin{itemize}
    \item lift Monoid inside Applicative
    \item promote any Monoid to an Applicative
    \end{itemize}
  \end{itemize}
\end{frame}

\begin{frame}[fragile]
  \frametitle{Monoids Inside Applicatives}
  \begin{itemize}
  \item for every \texttt{F[A]}, given \texttt{Applicative[F]} and \texttt{Monoid[A]}
  \item we can define
  \end{itemize}
  \vspace{5mm}
  \begin{minted}{scala}
def empty: F[A] = pure(Monoid[A].empty)

def combine(x: F[A], y: F[A]): F[A] =
  Applicative[F].map2(x, y)(Monoid[A].combine)
  \end{minted}
  \begin{itemize}
  \item that's Applicative $\rightarrow{}$ Monoid
  \item what about Monoid $\rightarrow{}$ Applicative
  \end{itemize}
\end{frame}

\begin{frame}[fragile]
  \frametitle{The Const Datatype}
  \begin{center}
    \inputminted[fontsize=\small]{scala}{snippets/const-def.scala}
  \end{center}
  \vspace{5mm}
  \begin{itemize}
  \item simple case class with phantom type parameter
  \item \ldots{} and one actual value
  \item define \texttt{Functor} and \texttt{Applicative}
  \end{itemize}
\end{frame}

\begin{frame}[fragile]
  \frametitle{The Const Datatype}
  \begin{center}
    \inputminted[fontsize=\small]{scala}{snippets/const-def.scala}
    \vfill
    \inputminted[fontsize=\small]{scala}{snippets/const-try-applicative.scala}
  \end{center}
\end{frame}

\begin{frame}[fragile]
  \frametitle{The Const Datatype}
  \begin{center}
    \inputminted[fontsize=\small]{scala}{snippets/const-def.scala}
    \vfill
    \inputminted[fontsize=\small, highlightlines={1,6,9}]{scala}{snippets/const-applicative.scala}
  \end{center}
\end{frame}

\begin{frame}
  \frametitle{The Const Datatype}
  \begin{itemize}
  \item the functor instance is a little strange
  \item but the applicative instance is super awesome
  \item remember \texttt{Option} lifting any \texttt{Semigroup}?
  \item allows you to lift any \texttt{Monoid} into an \texttt{Applicative}!
  \item that means we can still use everything we already have
  \item (nb: Const is a monoid isomorphism)
  \end{itemize}
\end{frame}

\begin{frame}
  \frametitle{Composition And Applicatives}
  \begin{itemize}
  \item remember \texttt{Monoid} composition? Nesting and tupling
  \item we get the same for \texttt{Applicative} (from e.g.\ \texttt{cats})
    \begin{itemize}
    \item Nested
    \item Tuple2K
    \end{itemize}
  \item why is that a big deal again?
  \end{itemize}
\end{frame}

\begin{frame}[fragile]
  \frametitle{Composition And Applicatives}
  \inputminted[fontsize=\small]{scala}{snippets/compose-applicative-monoid.scala}
  \inputminted[fontsize=\small]{scala}{snippets/compose-applicative-1.scala}
\end{frame}

\begin{frame}[fragile]
  \frametitle{Composition And Applicatives}
  \inputminted[fontsize=\small]{scala}{snippets/compose-applicative-2.scala}
\end{frame}

\begin{frame}[fragile]
  \frametitle{Composition And Applicatives}
  \begin{center}
    \begin{tikzpicture}[
      node distance=10mm,
      block/.style={rectangle, draw=black!50, anchor=west, font=\large\ttfamily}
      ]
      \node (io) { IO };

      \node (const) [fill=beamer@codeblue, below=of io.west, block, xshift=2mm] { Const(1) };
      \node (valid) [fill=beamer@centricgreen, right=of const.east, block] { Validated.Valid("...") };

      \node [draw=black!50, fit={(io) (const) (valid)}] {};
      \node [draw=black!50, fit={(const) (valid)}] {};
    \end{tikzpicture}
  \end{center}
\end{frame}

\begin{frame}[fragile]
  \frametitle{Composition And Applicatives}
  \begin{center}
    \begin{tikzpicture}[
      node distance=10mm,
      block/.style={rectangle, draw=black!50, anchor=west, font=\large\ttfamily}
      ]
      \node (tuple2k) { Tuple2K };

      \node (io1) [fill=beamer@codeblue, below=of tuple2k.west, block, xshift=2mm] { IO(<operation 1>) };
      \node (io2) [fill=beamer@centricgreen, right=of io1.east, block] { IO(<operation 2>) };

      \node [draw=black!50, fit={(io1) (io2) (tuple2k)}] {};
    \end{tikzpicture}
  \end{center}
\end{frame}

\begin{frame}[fragile]
  \frametitle{Composition And Applicatives}
  \begin{center}
    \begin{tikzpicture}[
      node distance=10mm,
      block/.style={rectangle, draw=black!50, anchor=west, font=\large\ttfamily}
      ]
      \node (tuple2k) { Tuple2K };

      \node (io1) [fill=beamer@codeblue, below=of tuple2k.west, block, xshift=2mm] { IO(<operation 1>) };
      \node (tuple2) [right=of io1.east] { Tuple2K };
      \node (io2) [fill=beamer@codeblue, below=of tuple2.west, block, xshift=2mm] { IO(<operation 2>) };
      \node (io3) [fill=beamer@codeblue, right=of io2.east, block] { IO(<operation 3>) };

      \node [draw=black!50, fit={(io1) (io2) (io3) (tuple2k)}] {};
      \node [draw=black!50, fit={(io2) (io3) (tuple2)}] {};
    \end{tikzpicture}
  \end{center}
\end{frame}

\begin{frame}
  \frametitle{Traversing With Applicatives}
  \begin{itemize}
  \item for monoids, \texttt{foldMap} is the essential operation
  \item for applicatives, change to \texttt{traverse}
  \item we can derive \texttt{foldMap} from \texttt{traverse}
  \end{itemize}
  \vspace{5mm}
  \inputminted[fontsize=\small]{scala}{snippets/traverse-def.scala}
\end{frame}

\begin{frame}
  \frametitle{Counting Words \textemdash{} With State}
  \begin{itemize}
  \item we can re-use the counting of chars and lines (and any monoidal metric)
  \item as seen, have to be more clever for words
  \item how to do this with applicatives?
  \item \textbf{State}, IO, ST, \ldots{}
  \end{itemize}
\end{frame}

\begin{frame}[fragile]
  \frametitle{Counting Words \textemdash{} With State}
  \inputminted[fontsize=\small]{scala}{snippets/count-words-state.scala}
\end{frame}

\begin{frame}[fragile]
  \frametitle{Counting Words \textemdash{} With State}
  \begin{center}
    \begin{tikzpicture}[
      node distance=10mm,
      block/.style={rectangle, draw=black!50, anchor=west, font=\large\ttfamily}
      ]
      \node (state) { State[Boolean, ?] };

      \node (const) [fill=beamer@codeblue, below=of state.west, block, xshift=2mm] { Const[Int, ?] };

      \node [draw=black!50, fit={(state) (const)}] {};
    \end{tikzpicture}
  \end{center}
\end{frame}

\begin{frame}[fragile]
  \frametitle{Counting Words \textemdash{} With State}
  \begin{itemize}
  \item putting it together
  \end{itemize}
  \vspace{5mm}
  \inputminted[fontsize=\small]{scala}{snippets/count-words-applicative.scala}
\end{frame}

\begin{frame}
  \frametitle{Monads Must Be Great!}
  \begin{itemize}
  \item if \texttt{Applicative} brings us so much power, how good must it be with \texttt{Monad}
  \item unfortunately, \texttt{Monad}s are not as good as they seem
  \item \texttt{Functor} and \texttt{Applicative} are \textbf{closed} under composition
  \item arbitrary nesting of \texttt{Functor} inside \texttt{Functor}
    / or \texttt{Applicative} inside \texttt{Applicative}
  \item \texttt{Monad} is \textbf{not} closed, not guaranteed to be a \texttt{Monad} at all!
  \item and Product (Tuple) does not work either
  \item see also ``The Essence Of The Iterator Pattern'', section 5.4
  \end{itemize}
\end{frame}

\section{Streaming}\label{sec:streaming}

\begin{frame}
  \frametitle{Streaming}
  \begin{center}
    \huge
    Streaming
  \end{center}
\end{frame}

\begin{frame}
  \frametitle{From Iteration To Streaming}
  \begin{itemize}
  \item there is no \texttt{traverse} for real streams?!
  \item it does \textbf{not make sense} for a stream (why?)
  \item but: do we need the actual \texttt{traverse}
  \item realization: \texttt{traverse\_} (\texttt{Foldable}) is enough!
  \end{itemize}
  \vspace{5mm}
  \inputminted[fontsize=\small]{scala}{snippets/foldable-traverse.scala}
\end{frame}

\begin{frame}
  \frametitle{Traverse and Foldable}
  \begin{itemize}
  \item so what's the difference between \texttt{traverse} and \texttt{traverse\_}
  \item the regular \texttt{traverse} keeps the whole structure!
  \item the \texttt{traverse\_} variant only sequences effects!
  \item can be implemented using a \texttt{fold}like method
  \item important: Applicative effect sequencing is \textbf{associative} as per the laws
  \item that means we can work on the stream in parallel
  \end{itemize}
\end{frame}

\begin{frame}[fragile]
  \frametitle{Implement traverse\_ for fs2}
  \begin{itemize}
  \item our framework needs the \texttt{traverse\_}
  \item can be implemented for anything that has a proper \texttt{fold}
  \item example with fs2 \texttt{Stream}
  \end{itemize}
  \vspace{5mm}
  \inputminted[fontsize=\small]{scala}{snippets/fs2-traverse-stream.scala}
\end{frame}

\begin{frame}[fragile]
  \frametitle{Counting Words With fs2}
  \inputminted[fontsize=\small]{scala}{snippets/fs2-count-words.scala}
\end{frame}

\begin{frame}
  \frametitle{Counting Words With Streams}
  \begin{itemize}
  \item suddenly we can use all the stream processing goodness
  \item we could chunk the input and process in parallel (but keep order)
  \item by carefully choosing the \texttt{Applicative}, operate with constant memory
  \item use existing connectors to read from DB, Kafka, etc.
  \item very flexible and almost for free!
  \end{itemize}
\end{frame}

\section{Conclusion}\label{sec:conclusion}

\begin{frame}
  \frametitle{Conclusion}
  \begin{itemize}
  \item flexible and \texttt{composable} way to cacluate metrics over text
  \item using \texttt{Monoid} and \texttt{Applicative}
  \item works with iteration and streaming
  \item Principle Of Least Power: sometimes Monads are overrated
  \end{itemize}
\end{frame}

\begin{frame}
  \frametitle{The End}
  \begin{itemize}
  \item \hyperref[sec:introduction]{Introduction}
  \item \hyperref[sec:monoids]{Monoids}
  \item \hyperref[sec:applicatives]{Applicatives}
  \item \hyperref[sec:streaming]{Streaming}
  \item \hyperref[sec:conclusion]{Conclusion}
  \end{itemize}
  \vfill
  \begin{center}
    \url{https://github.com/markus1189/flatmap-beautiful-composition}
  \end{center}
\end{frame}

\appendix{}

\begin{frame}
  \frametitle{Overview}
  \def\firstcircle{(0,0) ellipse (20mm and 5mm)}
  \def\secondcircle{(0,0) ellipse (25mm and 12mm)}
  \def\thirdcircle{(0,0) ellipse (35mm and 20mm)}
  \begin{center}
    \begin{tikzpicture}
      \begin{scope}[fill opacity=0.5]
        \fill[beamer@centricgreen!90] \firstcircle;
        \fill[beamer@codeblue!90] \secondcircle;
        \fill[red!5] \thirdcircle;
        \draw \firstcircle node  {$Semigroup$};
        \draw \secondcircle node [yshift=-8mm] {$Monoid$};
        \draw \thirdcircle node [yshift=-16mm] {$Applicative$};
      \end{scope}
    \end{tikzpicture}
  \end{center}
\end{frame}

\begin{frame}
  \frametitle{Abstraction}
  \smartdiagramset{set color list={beamer@centricgreen!80,beamer@codeblue!80,red!50}}
  \smartdiagram[priority descriptive diagram]{
    Semigroup,
    Monoid,
    Applicative
  }
\end{frame}

\begin{frame}
  \frametitle{Brand New: Selective Functors}
  \begin{itemize}
  \item like applicatives, we can use a product of selective functors
  \item example use case: find file containing a word
  \item applicative: cannot stop and iterates over everything, static analysis using free structure
  \item monad: can abort iteration after first match, no static analysis
  \item selective: abort early and also analysis via Free structure
  \end{itemize}
\end{frame}

\end{document}
